%preamble goes here
%parameters for beamer
\documentclass{beamer}
%\documentclass[svgnames,smaller]{beamer}
\usetheme{Rochester}
%\useoutertheme[subsection=false,footline=title]{miniframes}
\usecolortheme{fly}

%packages et al.
\usepackage[
    backend=bibtex,
    natbib=true,
    style=authoryear,
    url=false,
    doi=false,
    isbn=false,
]{biblatex}
\addbibresource{bibliography.bib}

%nice tables
\usepackage{booktabs}

%multiline
\usepackage{amsmath}

%appendix
\usepackage{appendix}
\usepackage{appendixnumberbeamer}

%titles et al
\title{The Origins of Mass Incarceration}

%author, affil
\author{Adaner Usmani}
\institute[Brown]
{
 Postdoctoral Fellow \\
 Watson Institute \\
 Brown University
}

%date
\date{November 2018}

%this makes it dark background, white text
\setbeamercovered{transparent} %this toggles on faded lists, w/ +/- command
\setbeamercolor{normal text}{fg=white,bg=black} %this is main text color

%this adds navigation symbols to the presentation
\addtobeamertemplate{navigation symbols}{}{%
    \usebeamerfont{footline}%
    \usebeamercolor[fg]{footline}% can be structure, if white
    \hspace{3em}%
    \insertframenumber/\inserttotalframenumber
}

%other formatting options
%(go through and add descriptions}
\setbeamercolor{item}{fg=red!50!yellow, bg=black} %items are orange
\setbeamercolor{structure}{fg=white}
\setbeamercolor{alerted text}{fg=red!50!yellow!} %alerted text is orange
\setbeamercolor{item projected}{use=item,fg=black,bg=item.fg!35}
\setbeamercolor*{palette primary}{use=structure,fg=structure.fg!95!black}
\setbeamercolor*{palette secondary}{use=structure,fg=structure.fg!95!black}
\setbeamercolor*{palette tertiary}{use=structure,fg=structure.fg!90!black}
\setbeamercolor*{palette quaternary}{use=structure,fg=structure.fg!95!black,bg=black!80}
\setbeamercolor*{framesubtitle}{fg=white}
\setbeamercolor*{block title}{parent=structure,bg=black!60}
\setbeamercolor*{block body}{fg=black,bg=black!10}
\setbeamercolor*{block title alerted}{parent=alerted text,bg=black!15}
\setbeamercolor*{block title example}{parent=example text,bg=black!15}

%for toc
\setbeamercolor{section number projected}{bg=black,fg=white}
\setbeamertemplate{sections/subsections in toc}[sections numbered]
\setbeamercolor{section in toc}{fg=white}
\setbeamerfont{subsection in toc}{size=\small}

%%%%%%%%%%%%%%%%%%%%%%%%%%%%%%%%%%%%%%

%define the claims as commands, which will make consistency easier
\newcommand{\claimone}{Due to the Civil Rights Movement, black Americans made gains}
\newcommand{\claimtwo}{Due to black gains, white Americans grew anxious}
\newcommand{\claimthree}{These racial anxieties led white Americans to fear crime}
\newcommand{\claimfour}{To exploit this fear of crime, white politicians proposed punitive policies}
\newcommand{\claimfive}{Due to white politicians' support of punitive policy, the rate of incarceration and policing increased}

%%%%%%%%%%%%%%%%%%%%%%%%%%%%%%%%%%%%%

\begin{document}

\begin{frame}
  \titlepage
\end{frame}

%%%%%%%%%%%%%%%%%%%%%%%%%%%%%%%%%%%%%%

%longitudinal puzzle
\section{The Puzzle}

\begin{frame}{}
\begin{center}
    \includegraphics[height=0.9\textheight,keepaspectratio]{fig_inmates_preturn.pdf}
\end{center}
\end{frame}

\begin{frame}{}
\begin{center}
   \includegraphics[height=0.9\textheight,keepaspectratio]{fig_inmates_punturn.pdf} 
\end{center}
\end{frame}

\begin{frame}{}
\begin{center}
    {\Large Why the punitive turn?}
\end{center}
\end{frame}

\begin{frame}{An Influential View}
\begin{itemize}
\item[] ``Law-and-order campaign appeals combined with a covert emphasis on the links between race and street crime used to overcome Republican electoral disadvantages seem to provide \alert{the most plausible explanations} for the rapid increase in U.S. imprisonment rates...''
\item[] {\footnotesize Jacobs and Jackson, "...A Review of Systematic Findings" (2010)}
\end{itemize}
\end{frame}

\begin{frame}{An Influential View}
\includegraphics[width=\textwidth,keepaspectratio]{fig_citerate.pdf}
\end{frame}

\begin{frame}{The Assertion}
\includegraphics[width=\textwidth,keepaspectratio]{fig_vp_spurious.png}
\end{frame}

\begin{frame}{Homicides Rose}
\includegraphics[width=\textwidth,keepaspectratio]{fig_crimerise1.pdf}
\end{frame}

\begin{frame}{Violence Rose}
\includegraphics[width=\textwidth,keepaspectratio]{fig_crimerise2.pdf}
\end{frame}

\begin{frame}{Crime Rose}
\includegraphics[width=\textwidth,keepaspectratio]{fig_crimerise3.pdf}
\end{frame}

\begin{frame}{Most Are Not Drug Offenders}
\begin{center}
    \includegraphics[width=3in,keepaspectratio]{fig_drugpriz.png}
\end{center}
\end{frame}

\begin{frame}{Fewer Are Low-Level Drug Offenders}
\begin{center}
    \includegraphics[width=3in,keepaspectratio]{fig_lldrugpriz.png}
\end{center}
\end{frame}

\begin{frame}{}
\begin{itemize}
    \item[] ``To omit violence from the analysis was to misunderstand the social inequality on which mass incarceration rests.'' \\
    \item[] Bruce Western, \textit{Homeward} (2018)
\end{itemize}
\end{frame}

\begin{frame}{\textbf{Not} The Argument}
\includegraphics[width=\textwidth,keepaspectratio]{fig_vp_naive.png}
\end{frame}

\begin{frame}{The Developing View}
\includegraphics[width=\textwidth,keepaspectratio]{fig_vp_moderated_empty.png}
\end{frame}

\begin{frame}{My Argument}
\includegraphics[width=\textwidth,keepaspectratio]{fig_vp_moderated_sd.png}
\end{frame}

%%%%%%%%%%%%%%%%%%%%%%%%%%%%%%%%%%%%%%
%%%%%%%%%%%%%%%%%%%%%%%%%%%%%%%%%%%%%%

\begin{frame}{Roadmap}
    \tableofcontents
\end{frame}

%from here on, show TOC
\AtBeginSection[]
{
  \begin{frame}
    \frametitle{Roadmap}
    \tableofcontents[currentsection]
  \end{frame}
}

%%%%%%%%%%%%%%%%%%%%%%%%%%%%%%%%%%%%%%

\section{The Influential View}

\begin{frame}{Five Claims} \pause
\begin{itemize}[<+->]
    \item[1.] \claimone
    \item[2.] \claimtwo
    \item[3.] \claimthree
    \item[4.] \claimfour
    \item[5.] \claimfive
\end{itemize}
\end{frame}

\begin{frame}{Five Claims}
\begin{itemize}
    \item[1.] \claimone
    \item[2.] \claimtwo
    \alert{\item[3.] \claimthree
    \item[4.] \claimfour
    \item[5.] \claimfive}
\end{itemize}
\end{frame}

%%%%%%%%%%%%%%%%%%%%%%%%%%%%%%%%%%%%%%

\section{New Evidence}

\newcommand{\claimthreeq}{Why did public opinion change?}
\newcommand{\claimfourq}{Why did politicians turn punitive?}
\newcommand{\claimfiveq}{Why did prisons and police grow?}

\subsection{\claimthreeq}

\begin{frame}{}
\begin{center}
    {\Large \claimthreeq}
\end{center}
\end{frame}

\begin{frame}{\claimthreeq} 
\begin{itemize}[<+->]
    \item[] \alert{Data:} 300,000 respondents to 39 different questions over 176 different public opinion surveys about crime and punishment, 1955-2014
    \item[] \alert{Method:} 
    \begin{itemize}
        \item Summarize levels and white-black gap
        \item Estimate trends by race, 1955-2014
    \end{itemize}
\end{itemize}
\end{frame}

\begin{frame}{\claimthreeq} 
\begin{itemize}
    \item[] \alert{Data:} 300,000 respondents to 39 different questions over 176 different public opinion surveys about crime and punishment, 1955-2014
    \item[] \alert{Method:} 
    \begin{itemize}
        \item Summarize levels and white-black gap
        \item \alert{Estimate trends by race, 1955-2014}
    \end{itemize}
\end{itemize}
\end{frame}

\begin{frame}{Two Dimensions}
\begin{itemize}
    \item[] \alert{Punitiveness, e.g.}
    \begin{itemize}
        \item In general, do you think the courts in this area deal too harshly or not harshly enough with criminals? 
        \item Do you favor or oppose the death penalty for persons convicted of murder?
    \end{itemize}
    \item[] \alert{Anxiety, e.g.}
        \begin{itemize}
        \item ...I'd like you to tell me whether you think we're spending too much money, too little money, or about the right amount on halting the rising crime rate. 
        \item Is being a victim of crime something you personally worry about?
    \end{itemize}
\end{itemize}
\end{frame}

\begin{frame}{The Model} 
\begin{itemize}
    \item[] \alert{Challenge:} Different questions in different years
    \item[] \alert{Solution:} Multilevel models with (raceX)question-level random effects
\end{itemize}
\end{frame}

\begin{frame}{As if Nonprobability Sampling} 
\begin{itemize}
    \item[] \alert{Challenge:} Sample weights are difficult to harmonize
    \item[] \alert{Solution:} Predictions for $sex \times race \times ed \times age \times year$ cells, weighted by share of population in year $t$ to obtain representative estimate
\end{itemize}
\end{frame}

\begin{frame}{\claimthreeq}
\begin{center}
    \includegraphics[height=0.9\textheight,keepaspectratio]{fig_trends_pun_influential.pdf}
\end{center}
\end{frame}

\begin{frame}{\claimthreeq}
\begin{center}
    \includegraphics[height=0.9\textheight,keepaspectratio]{fig_trends_anx_influential.pdf}
\end{center}
\end{frame}

\begin{frame}{\claimthreeq}
\begin{center}
    \includegraphics[height=0.9\textheight,keepaspectratio]{fig_trends_pun_actual.pdf}
\end{center}
\end{frame}

\begin{frame}{\claimthreeq}
\begin{center}
    \includegraphics[height=0.9\textheight,keepaspectratio]{fig_trends_anx_actual.pdf}
\end{center}
\end{frame}

\begin{frame}{Is \textit{white} punitiveness driven by revanchism?} 
\begin{center}
    \includegraphics[height=0.9\textheight,keepaspectratio]{fig_white.pdf}
\end{center}
\end{frame}

\begin{frame}{Is \textit{white} punitiveness driven by revanchism?} 
\begin{center}
    \includegraphics[height=0.9\textheight,keepaspectratio]{fig_whiteblack.pdf}
\end{center}
\end{frame}

\begin{frame}{Is \textit{white} punitiveness driven by revanchism?} 
\begin{center}
    \includegraphics[height=0.9\textheight,keepaspectratio]{fig_whiteblackprotest.pdf}
\end{center}
\end{frame}

\begin{frame}{Is \textit{white} punitiveness driven by revanchism?} 
\begin{center}
    \includegraphics[height=0.9\textheight,keepaspectratio]{fig_whiteblackcrime.pdf}
\end{center}
\end{frame}

\begin{frame}{}
\begin{center}
    {\Large \claimfourq}
\end{center}
\end{frame}

\subsection{\claimfourq}
\begin{frame}{\claimfourq}
\begin{itemize}[<+->]
    \item[] \alert{Data:} Roll call data from the House of Representatives, 30 punitive bills, 1968-2015
    \item[] \alert{Method:} 
    \begin{itemize}
        \item Summarize level and white-black gap for each punitive bill
        \item Summarize trends in punitive voting, 1968-2015
    \end{itemize}
\end{itemize}
\end{frame}

\begin{frame}{\claimfourq}
\begin{itemize}
    \item[] \alert{Data:} Roll call data from the House of Representatives, 30 punitive bills, 1968-2015
    \item[] \alert{Method:} 
    \begin{itemize}
        \item Summarize level and white-black gap for each punitive bill
        \item \alert{Summarize trends in punitive voting, 1968-2015}
    \end{itemize}
\end{itemize}
\end{frame}

\begin{frame}{\claimfourq}
\begin{center}
    \includegraphics[height=0.9\textheight,keepaspectratio]{fig_vtrends_influential_view.pdf}
\end{center}
\end{frame}

\begin{frame}{\claimfourq}
\begin{center}
    \includegraphics[height=0.9\textheight,keepaspectratio]{fig_vtrends_estimated.pdf}
\end{center}
\end{frame}

\begin{frame}{Was this representation or selling-out?}
\begin{itemize}[<+->]
    \item[] \alert{Why support Clinton's 1994 Crime Bill?}
    \begin{itemize}
        \item 63\% of African Americans supported, only 20\% opposed
        \item Many objections, but``[t]he crime bill's promise of more police, more prisons and more money for crime prevention was too important to jeopardize by holding out for the racial-justice provision.'' (Alan Wheat, D-MO)
    \end{itemize}
\end{itemize}
\end{frame}

\begin{frame}{}
\begin{center}
    {\Large \claimfiveq}
\end{center}
\end{frame}

\subsection{\claimfiveq}
\begin{frame}{\claimfiveq}
\begin{itemize}[<+->]
    \item[] \alert{Data:} Records of black political representation from Joint Center for Political and Economic Studies
    \item[] \alert{Method:}
    \begin{itemize}
        \item Exploit redistricting after 1990 census to estimate impact on incarceration and policing, 1986-1996
        \item Within-state correlations between black political representation and incarceration/policing, 1970-2010
    \end{itemize}
\end{itemize}
\end{frame}

\begin{frame}{\claimfiveq}
\begin{itemize}
    %\item[] \alert{Claim:} \claimfive
    \item[] \alert{Data:} Records of black political representation from Joint Center for Political and Economic Studies
    \item[] \alert{Method:}
    \begin{itemize}
        \item \alert{Exploit redistricting after 1990 census to estimate impact on incarceration and policing, 1986-1996}
        \item Within-state correlations between black political representation and incarceration/policing, 1970-2010
    \end{itemize}
\end{itemize}
\end{frame}

\begin{frame}{A Natural Experiment} 
\begin{center}
    \includegraphics[height=0.9\textheight,keepaspectratio]{fig_beodd.pdf}
\end{center}
\end{frame}

\begin{frame}{\claimfiveq}
\begin{center}
    \includegraphics[width=\textwidth,keepaspectratio]{fig_panelest_influential_view.pdf}
\end{center}
\end{frame}

\begin{frame}{\claimfiveq}
\begin{center}
    \includegraphics[width=\textwidth,keepaspectratio]{fig_panelest_estimated_redistricting.pdf}
\end{center}
\end{frame}

\begin{frame}{Was this just racial threat?}
\begin{center}
    \includegraphics[width=\textwidth,keepaspectratio]{fig_ddwelfare_rathreat.pdf}
\end{center}
\end{frame}

\begin{frame}{Was this just racial threat?}
\begin{center}
    \includegraphics[width=\textwidth,keepaspectratio]{fig_ddwelfare.pdf}
\end{center}
\end{frame}

\subsection{Interpretation}

\begin{frame}{The Influential Account}
\begin{itemize}
    \item[3.] \claimthree
    \item[4.] \claimfour
    \item[5.] \claimfive
\end{itemize}
\end{frame}

\begin{frame}{My Account}
\begin{itemize}
    \item[3.]<+> The rise in violence drove the rise in public punitiveness
    \item[4.]<+> As a result, politicians supported punitive policies in greater numbers
    \item[5.]<+> White \textit{and} black elected officials were responsible for the increase in the rate of incarceration and policing
\end{itemize}
\end{frame}

\begin{frame}{My Interpretation}
\begin{center}
    {\Large The rise in violence mattered.}
\end{center}
\end{frame}

%%%%%%%%%%%%%%%%%%%%%%%%%%%%%%%%%%%%%%

\section{Refining the Question}

\begin{frame}{}
\begin{center}
    {\Large Why the punitive turn?}
\end{center}
\end{frame}

\begin{frame}{Penal Policy}
\begin{center}
    \includegraphics[width=\textwidth,keepaspectratio]{fig_unidimensional_empty.png}
\end{center}
\end{frame}

\begin{frame}{What Is To Be Done?}
\begin{itemize}
\item[] ``The ecological concentration of ghetto poverty, racial segregation, residential mobility and population turnover, family disruption, and... local social organization... are fruitful areas of future inquiry... \alert{Our framework suggests the need to take a renewed look at social policies that focus on prevention.} We do not need more after-the-fact (reactive) approaches.''
\item[] Sampson and Wilson, ``Towards a Theory of Race, Crime, and Urban Inequality'' (1995)
\end{itemize}
\end{frame}

\begin{frame}{Penal Policy \textit{and} Social Policy}
\begin{center}
    \includegraphics[width=\textwidth,keepaspectratio]{fig_bidimensional_empty.png}
\end{center}
\end{frame}

\begin{frame}{Four Quadrants}
\begin{center}
    \includegraphics[width=\textwidth,keepaspectratio]{fig_bidimensional_quadrants.png}
\end{center}
\end{frame}

\begin{frame}{Why Quadrant III?}
\begin{center}
    \includegraphics[width=\textwidth,keepaspectratio]{fig_bidimensional_quadrants_q3.png}
\end{center}
\end{frame}

\begin{frame}{}
\begin{center}
    {\Large Why the punitive turn?}
\end{center}
\end{frame}

\begin{frame}{}
\begin{center}
    {\Large \alert{Why did America fight violence with penal policy?}}
\end{center}
\end{frame}

\begin{frame}{Penal Policy vs. Social Policy}
    \includegraphics[width=\textwidth,keepaspectratio]{fig_penalvssocial1.png}
\end{frame}

\begin{frame}{Penal Policy vs. Social Policy}
    \includegraphics[width=\textwidth,keepaspectratio]{fig_penalvssocial2.png}
\end{frame}

\begin{frame}{Penal Policy vs. Social Policy}
    \includegraphics[width=\textwidth,keepaspectratio]{fig_penalvssocial4.png}
\end{frame}

\begin{frame}{Cost of Quadrant III}
\begin{center}
    \includegraphics[width=\textwidth,keepaspectratio]{fig_bidimensional_quadrants_priced_q3.png}
\end{center}
\end{frame}

\begin{frame}{Abolition vs. Punitiveness}
\begin{center}
    \includegraphics[width=\textwidth,keepaspectratio]{fig_bidimensional_quadrants_priced_abolition.png}
\end{center}
\end{frame}

\begin{frame}{Laissez-Faire vs. Root Causes}
\begin{center}
    \includegraphics[width=\textwidth,keepaspectratio]{fig_bidimensional_quadrants_priced_rootcauses.png}
\end{center}
\end{frame}

\begin{frame}{Cost in Four Quadrants}
\begin{center}
    \includegraphics[width=\textwidth,keepaspectratio]{fig_bidimensional_quadrants_priced_allqs.png}
\end{center}
\end{frame}

\begin{frame}{}
\begin{center}
    {\Large Why did America fight violence with penal policy?}
\end{center}
\end{frame}

\begin{frame}{}
\begin{center}
    {\Large \alert{Why did America fight violence on the cheap?}}
\end{center}
\end{frame}

\begin{frame}{Some Deductions}
\begin{itemize}
    \item[1.] The state requires economic resources
    \item[2.] Because the rich have greater capacities to disrupt the economy, they have greater leverage over the state
    \item[3.] But the poor sometimes gain capacities to disrupt, and so they gain leverage over the state
    \item[4.] As the poor gain leverage, redistribution from rich to poor becomes more likely
\end{itemize}
\begin{flushright}
{\scriptsize ``Democracy and the Class Struggle'', \textit{American Journal of Sociology}, 124(3):1-41}
\end{flushright}
\end{frame}

\begin{frame}{A Conclusion}
\begin{center}
    {\Large Redistribution = \alert{f(Balance of Disruptive Capacities)}}
\end{center}
%\begin{flushright}
%{\scriptsize ``Democracy and the Class Struggle'', \textit{American Journal of Sociology}, 124(3):1-41}
%\end{flushright}
\end{frame}

\begin{frame}{}
\begin{itemize}
    \item[] ``Those who against the public weal have power cannot be expected to yield save to superior power.'' \\
    \item[] W.E.B Du Bois, \textit{Black Reconstruction} (1935)
\end{itemize}
\end{frame}

\begin{frame}{}
\begin{center}
    {\Large Why did America fight violence on the cheap?}
\end{center}
\end{frame}

\begin{frame}{A Hypothesis}
\begin{center}
    {\Large \alert{The poor lacked the capacity to win social policy.}}
\end{center}
\end{frame}

\begin{frame}{Longstanding}
\begin{itemize}[<+->]
    \item[] \alert{Balance of capacities is unfavorable}
        \begin{itemize}
            \item A working-class divided by race
            \item Institutions that enable elite veto
            \item[] (\textit{And behind both, slavery..)}
        \end{itemize}
\end{itemize}
\end{frame}

\begin{frame}{Conjunctural}
\begin{itemize}[<+->]
    \item[] \alert{Capacities of the powerless declined}
    \begin{itemize}
        \item[] The Civil Rights movement faded, the labor movement ossified
    \end{itemize}
    \item[] \alert{Leverage of the powerful increased}
    \begin{itemize}
        \item[] The economy sputtered, Vietnam drained the Federal budget
    \end{itemize}
\end{itemize}
\end{frame}

%%%%%%%%%%%%%%%%%%%%%%%%%%%%%%%%%%%%%%

\section{In Sum}

\begin{frame}{The Influential View}
\includegraphics[width=\textwidth,keepaspectratio]{fig_vp_spurious.png}
\end{frame}

\begin{frame}{The Developing View}
\begin{center}
        \includegraphics[width=\textwidth,keepaspectratio]{fig_vp_moderated_empty.png}
\end{center}
\end{frame}

\begin{frame}{A Tale of Two Exceptions}
\begin{center}
        \includegraphics[width=\textwidth,keepaspectratio]{fig_vp_moderated_sd.png}
\end{center}
\end{frame}

\begin{frame}{A Tale of Two Exceptions}
\begin{center}
        \includegraphics[width=\textwidth,keepaspectratio]{fig_vp_confounded.png}
\end{center}
\end{frame}

%A research agenda?
\begin{frame}{A Research Agenda in Six Claims}
    \begin{itemize}
        \item[] \alert{Punitiveness}
        \begin{itemize}
        \item[1.] Risk of Victimization $\rightarrow$ Public Punitiveness(/Anxiety)
        \item[2.] Public Punitiveness $\rightarrow$ Punitiveness of Politicians
        \end{itemize}
        \item[] \alert{Policy}
        \begin{itemize}
        \item[3.] Balance of Power $\rightarrow$ More Social Policy
        \item[4.] More Social Policy $\rightarrow$ Less Penal Policy
        \end{itemize}
        \item[] \alert{Outcomes}
        \begin{itemize}
        \item[5.] More Social Policy $\rightarrow$ Less Violence
        \item[6.] Mass Incarceration $\rightarrow$ Macrosociological Outcomes
        \end{itemize}
    \end{itemize}
\end{frame}

\begin{frame}{Ideas}
    \begin{itemize}
        \item[1.] ML models estimating risk of victimization using $sex \times race \times ed \times age \times year$ info, enter as predictor in $p(anxiety|punitive|mistrustful)$
        \item[1.,2.] Survey experiments to prime respondents to risk of victimization, some given social policy options
        \item[3.] Analysis of utterances in Congress to place legislators on penal/social 2x2
        \item[3.] Bartik-like instruments for labor capacity using industry-level growth rates and county $\times$ industry employment shares
        \item[4., 5.] Trade shocks to labor markets, to simulate `as if' social policy
        \item[6.] Bartik-like instruments for growth in incarceration using $sex \times race \times ed \times age \times year$ institutionalization rates and county-level demographic shares
    \end{itemize}
\end{frame}

\begin{frame}{}
\begin{center}
    {\Large adaner\_usmani@brown.edu} \\
    {\large http://bit.ly/au\_jobtalkrepo}
\end{center}
\end{frame}

%%%%%%%%%%%%%%%%%%%%%%%%%%%%%%%%%%%%%%
%%%%%%%%%%%%%%%%%%%%%%%%%%%%%%%%%%%%%%

\appendix

\begin{frame}{}
\begin{center}
    {\Large Appendix}
\end{center}
\end{frame}

\begin{frame}{Crime and Punishment, Stock}
\includegraphics[width=\textwidth,keepaspectratio]{fig_crimepun.pdf}
\end{frame}

\begin{frame}{Crime and Punishment, Flow}
\includegraphics[width=\textwidth,keepaspectratio]{fig_crimeDpun.pdf}
\end{frame}

\begin{frame}{Admissions Per Crime}
\begin{center}
    \includegraphics[width=\textwidth,keepaspectratio]{fig_admissionspercrime.pdf}
\end{center}
\end{frame}

\begin{frame}{The Influential View?}
\includegraphics[width=\textwidth,keepaspectratio]{fig_vp_moderated_racism.png}
\end{frame}

\begin{frame}{The Developing View}
\begin{itemize}
    \item Racism \parencite{FormanJr2017}
    \item Culture, Legal History \parencite{Whitman2005,Garland2010}
    \item Media \parencite{Enns2016}
    \item Democracy in CJ \parencite{Savelsberg1994,Lacey2008,Garland2010,Lacey2015}
    \item Discretion in CJ \parencite{Pfaff2017}
    \item Federalism \parencite{Miller2008,Miller2016}
    \item Liberal Market Economy \parencite{Lacey2015,Garland2018}
\end{itemize}
\end{frame}

\begin{frame}{Racial Threat}
\begin{itemize}
    \item[1.] \claimone
    \item[2.] \alert{Due to black gains,} \textcolor{black!80}{white Americans grew anxious
    \item[3.] \claimthree
    \item[4.] \claimfour
    \item[5.] Due to white politicians' support of punitive policy,} \alert{the rate of incarceration and policing increased}
\end{itemize}
\end{frame}

\begin{frame}{Evidence via Racial Threat}
\begin{itemize}
\item[2-5.] Due to Af-Am gains, ... the rate of incarceration and policing increased
\begin{itemize}
    \item \%black correlated with punitive outcomes across and within cities/states
\end{itemize}
\end{itemize}
\end{frame}

\begin{frame}{Evidence for Claim 3}
\begin{itemize}
    \item[3.] \claimthree
    \begin{itemize}
        \item Over-time: The rise in punitiveness is correlated with the success of the Civil Rights movement
        \item Cross-individual: More racially-anxious people are more punitive
    \end{itemize}
\end{itemize}
\end{frame}

\begin{frame}{Evidence for Claim 4}
\begin{itemize}
    \item[4.] \claimfour
        \begin{itemize}
        \item Over-time: The dawn of punitive policy is correlated with the rise of punitive opinion
        \item Cross-politician: Those closest to the abandoned white voter are also the most punitive
    \end{itemize}
\end{itemize}
\end{frame}

\begin{frame}{Evidence for Claim 5}
\begin{itemize}
    \item[5.] \claimfive
    \begin{itemize}
        \item Over-time, national-level: Richard Nixon and Ronald Reagan played a leading role
        \item Cross-state/over-time, state-level: More Republican, more incarceration
    \end{itemize}
\end{itemize}
\end{frame}

\begin{frame}{Decomposition, Incarceration Rate}
Incarceration Rate = \\
Behavior \times \\ 
p(Crime$\mid$Behavior) \times \\ 
p(Arrest$\mid$Crime) \times \\ 
p(Charge$\mid$Arrest) \times \\ 
p(Conviction$\mid$Charge) \times \\ 
E(Time Served)
\end{frame}

\begin{frame}{Decomposition, Influential View}
Incarceration Rate = \\
Behavior \times \\ 
\alert{p(Crime$|$Behavior) \times \\ 
p(Arrest$|$Crime) \times \\ 
p(Charge$|$Arrest) \times \\ 
p(Conviction$|$Charge) \times \\ 
E(Time Served)}
\end{frame}

\begin{frame}{Correlates of Public Opinion}
\begin{center}
    \includegraphics[height=0.9\textheight,keepaspectratio]{fig_violencecorrs.pdf}
\end{center}
\end{frame}

\begin{frame}{Trends by Race, Subtrends Model}
\begin{center}
    \includegraphics[width=\textwidth,keepaspectratio]{fig_trends_exp.pdf}
\end{center}
\end{frame}

\begin{frame}{White Southerners vs. White Non-Southerners, Punitiveness}
\begin{center}
    \includegraphics[height=0.9\textheight,keepaspectratio]{fig_trends_whitesouth_p(punitive).pdf}
\end{center}
\end{frame}

\begin{frame}{White Southerners vs. White Non-Southerners, Anxiety}
\begin{center}
    \includegraphics[height=0.9\textheight,keepaspectratio]{fig_trends_whitesouth_p(anxiety).pdf}
\end{center}
\end{frame}

\begin{frame}{Is black punitiveness driven by elites?}
\begin{center}
    \includegraphics[height=0.9\textheight,keepaspectratio]{fig_trends_blacked_p(punitive).pdf}
\end{center}
\end{frame}

\begin{frame}{Is black anxiety driven by elites?}
\begin{center}
    \includegraphics[height=0.9\textheight,keepaspectratio]{fig_trends_blacked_p(anxiety).pdf}
\end{center}
\end{frame}

%public opinion model
\begin{frame}{Public Opinion Model, Standard}
\begin{equation*}
\label{standard}
\begin{split}
\Pr ( y_i = 1 ) =
logit^{-1} ( 
\beta^{0} + 
\beta^{race} RACE_{i} +
\beta^{sex} SEX_{i} + 
\beta^{ed} ED_{i} + \\
\beta_{age} AGE_{i} + 
\alpha_{q[i]}^{question} + 
\alpha_{s[i]}^{division} + 
\alpha_{t[i]}^{year} + 
\alpha_{j[i],k[i]}^{race.sex} + \\
\alpha_{j[i],m[i]}^{race.age} + 
\alpha_{j[i],q[i]}^{race.question} 
\alpha_{j[i],l[i]}^{race.ed} +
\alpha_{j[i],t[i]}^{race.year} + 
\alpha_{j[i],s[i]}^{race.division}
)
\end{split}
\end{equation*}
\end{frame}

%public opinion model
\begin{frame}{Public Opinion Model, Subtrends}
\begin{equation*}
\label{subtrends}
\begin{split}
\Pr ( y_i = 1 ) =
logit^{-1} ( 
\beta^{0} + 
\beta^{race} RACE_{i} +
\beta^{sex} SEX_{i} + 
\beta^{ed} ED_{i} + \\
\beta_{age} AGE_{i} + 
\alpha_{q[i]}^{question} + 
\alpha_{s[i]}^{division} + 
\alpha_{t[i]}^{year} + 
\alpha_{j[i],k[i]}^{race.sex} + \\
\alpha_{j[i],m[i]}^{race.age} + 
\alpha_{j[i],q[i]}^{race.question} +
\alpha_{j[i],l[i],t[i]}^{race.ed.year} + 
\alpha_{j[i],s[i],t[i]}^{race.division.year}
)
\end{split}
\end{equation*}
\end{frame}


%diff in diff equation
\begin{frame}{Differences-in-Differences Model}
\begin{equation*}
\label{dd_eq}
DV_{st} =  
(RD_{s} \times PD_{t}) \theta +
x'_{st-1} \beta + 
\delta_{s} + 
\mu_{t} + 
\epsilon_{st}
\end{equation*}
\begin{itemize}
    \item[] \alert{Controls:} Violent Crime, Partisanship, \% Black, GDP per capita, Growth Rate, Tax Collections, Gini Coefficient, Enns' Punitiveness Index
\end{itemize}
\end{frame}

\begin{frame}{\claimfiveq}
\begin{center}
    \includegraphics[width=\textwidth,keepaspectratio]{fig_panelest_estimated_correlations.pdf}
\end{center}
\end{frame}

\begin{frame}{Robustness Tests, Panel Regressions}
\begin{center}
    \includegraphics[width=\textwidth,keepaspectratio]{fig_robests.pdf}
\end{center}
\end{frame}

%where people fit
\begin{frame}{Penal Policy and Social Policy}
\begin{center}
    \includegraphics[width=\textwidth,keepaspectratio]{fig_bidimensional_stylized.png}
\end{center}
\end{frame}

\begin{frame}{Cost in Four Quadrants}
\begin{itemize}
    \item[] Penal Policy, Punitiveness = \$250 billion 
    \item[] Penal Policy, Abolition = \$50 billion 
    \item[] Social Policy, Laissez-Faire = \$1000 billion
    \item[] Social Policy, Root Causes = \$2000 billion
\end{itemize}
\end{frame}

\begin{frame}{Cost in Four Quadrants}
\begin{center}
    \includegraphics[width=\textwidth,keepaspectratio]{fig_bidimensional_quadrants_priced.png}
\end{center}
\end{frame}

%four puzzles
\begin{frame}{The Four Puzzles}
\begin{itemize}
    \item[1.] \alert{Longitudinal:} Why the punitive turn?
    \item[2.] \alert{Crossnational:} Why the US? 
    \item[3.] \alert{Subnational:} Why some parts but not others? 
    \item[4.] \alert{Cross-individual:} Why some people but not other people?
\end{itemize}
\end{frame}

\end{document}

